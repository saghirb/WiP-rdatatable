\documentclass[a4paper,9pt,twocolumn,twoside,printwatermark=false]{pinp}

%% Some pieces required from the pandoc template
\providecommand{\tightlist}{%
  \setlength{\itemsep}{0pt}\setlength{\parskip}{0pt}}

% Use the lineno option to display guide line numbers if required.
% Note that the use of elements such as single-column equations
% may affect the guide line number alignment.

\usepackage[T1]{fontenc}
\usepackage[utf8]{inputenc}

% The geometry package layout settings need to be set here...
\geometry{layoutsize={0.95588\paperwidth,0.98864\paperheight},%
          layouthoffset=0.02206\paperwidth,%
		  layoutvoffset=0.00568\paperheight}

\definecolor{pinpblue}{HTML}{185FAF}  % imagecolorpicker on blue for new R logo
\definecolor{pnasbluetext}{RGB}{101,0,0} %



\title{Women in Parliament -- data.table}

\author[]{Saghir Bashir}


\setcounter{secnumdepth}{3}

% Please give the surname of the lead author for the running footer
\leadauthor{\href{https://creativecommons.org/licenses/by-sa/4.0/}{CC BY SA}
\href{https://ilustat.com/}{ilustat} \(\bullet\)
\href{mailto:info@ilustat.com}{\nolinkurl{info@ilustat.com}}}

% Keywords are not mandatory, but authors are strongly encouraged to provide them. If provided, please include two to five keywords, separated by the pipe symbol, e.g:
 \keywords{  Women in Parliament |  World Bank Indicator |  data.table |  Tinyverse  }  

\begin{abstract}
We will use the World Bank's indicator data for ``Women in Parliament''
as a case study when working with the data.table R package. We will
guide you through the geographical and time trends for the percentage of
women in national parliaments. We will start by learning about and
understanding the raw data, which we will then process (``wrangle'') in
preparation for some exploratory analysis.
\end{abstract}

\dates{This version was compiled on \today} 

% initially we use doi so keep for backwards compatibility
% new name is doi_footer
\doifooter{Learn more at \url{https://ilustat.com/resources/}}

\pinpfootercontents{Women in Parliament -- data.table}

\begin{document}

% Optional adjustment to line up main text (after abstract) of first page with line numbers, when using both lineno and twocolumn options.
% You should only change this length when you've finalised the article contents.
\verticaladjustment{-2pt}

\maketitle
\thispagestyle{firststyle}
\ifthenelse{\boolean{shortarticle}}{\ifthenelse{\boolean{singlecolumn}}{\abscontentformatted}{\abscontent}}{}

% If your first paragraph (i.e. with the \dropcap) contains a list environment (quote, quotation, theorem, definition, enumerate, itemize...), the line after the list may have some extra indentation. If this is the case, add \parshape=0 to the end of the list environment.


\hypertarget{preface}{%
\section{Preface}\label{preface}}

We present a real-life case study for the data.table\footnote{For more
  information on the \texttt{data.table} package see
  \url{http://r-datatable.com/}.} package using the World Bank's ``Women
in Parliament'' indicator data. To get the most out of this case-study
guide, repeat the examples and do the exercises whilst reading it.

\hypertarget{guide-materials}{%
\subsection{Guide materials}\label{guide-materials}}

You can download materials for this guide from this link:

\begin{itemize}
\tightlist
\item
  \url{https://ilustat.com/shared/WiP-rdatatable.zip}
\end{itemize}

Unzip the file, which contains the data, this guide and an R script
exercise file. We advise you to work with ``\texttt{WiP-Exercise.R}''
file to follow the examples and do the exercises. If you are using
RStudio, you can double click on ``\texttt{WiP-dt.Rproj}'' to get
started.

\hypertarget{objectives}{%
\section{Objectives}\label{objectives}}

\emph{Explore the geographical and time trends for the percentage of
women\footnote{The objective could be termed neutrally as ``gender
  trends'' but we will keep it per the World Bank data.} in national
parliaments.}

\hypertarget{understanding-the-data}{%
\section{Understanding the Data}\label{understanding-the-data}}

\hypertarget{the-world-bank-data}{%
\subsection{The World Bank Data}\label{the-world-bank-data}}

The raw data for \emph{``Proportion of seats held by women in national
parliaments''} includes the percentage of women in parliament
(\emph{``single or lower parliamentary chambers only''}) by country
(region) and year. It can be downloaded from:\footnote{The
  \texttt{wbstats} R package
  (\url{https://cran.r-project.org/web/packages/wbstats/}) gives access
  to a ``tidier'' version of the World Bank indicator data.}

\begin{itemize}
\tightlist
\item
  \url{https://data.worldbank.org/indicator/SG.GEN.PARL.ZS}
\end{itemize}

As part of its ``open data'' mission the World Bank offers \emph{``free
and open access to global development data''} kindly licensed under the
``Creative Commons Attribution 4.0 (CC-BY 4.0)''.\footnote{\url{https://datacatalog.worldbank.org/public-licenses\#cc-by}.}

\hypertarget{source-data}{%
\subsection{Source Data}\label{source-data}}

The data originates from the ``Inter-Parliamentary Union''
(IPU)\footnote{Inter-Parliamentary Union: \url{https://www.ipu.org/}.}
which provides an \emph{\emph{``Archive of statistical data on the
percentage of women in national parliaments''}} going back to 1997 on a
monthly basis:

\begin{itemize}
\tightlist
\item
  \url{http://archive.ipu.org/wmn-e/classif-arc.htm}
\end{itemize}

The World Bank data is for ``single or lower parliamentary chambers
only'', while the IPU also presents data for ``Upper Houses or
Senates''. Moreover, the IPU provides the actual numbers used to
calculate the percentages (which the World Bank does not).

\hypertarget{data-limitations}{%
\subsection{Data limitations}\label{data-limitations}}

Take caution when interpreting these data, as parliamentary systems vary
from country to country, and in some cases over time. Some of the issues
to consider include:

\begin{itemize}
\tightlist
\item
  Who has, and who does not have, the right to become a Member of
  Parliament (MP)?
\item
  How does someone become an MP? Through democratic elections? How is
  ``democratic election'' defined?
\item
  What is the real power of MPs and their parliament? Can MPs make a
  difference?
\end{itemize}

\hypertarget{data-definitions-assumptions}{%
\subsection{Data definitions \&
assumptions}\label{data-definitions-assumptions}}

\hypertarget{women}{%
\subsubsection{``Women''}\label{women}}

The definition for ``women'' is not given, so we will assume that it
refers to a binary classification for gender (sex).

\hypertarget{country-region}{%
\subsubsection{``Country (Region)''}\label{country-region}}

The definition of countries and regions can change over time.
(e.g.~formation of new countries after conflicts, new member states
joining a pre-existing collective). How are these changes reflected in
the data? How do they affect the interpretation?

\hypertarget{pro-tip}{%
\subsection{Pro tip}\label{pro-tip}}

Understand the limitations of your data before anybody else points them
out to you.

\hypertarget{about-the-data-file}{%
\section{About the data file}\label{about-the-data-file}}

The data is stored in a file called:

\begin{itemize}
\tightlist
\item
  \texttt{API\_SG.GEN.PARL.ZS\_DS2\_en\_csv\_v2\_10515251.csv}
\end{itemize}

To simplify things we have copied it to \texttt{WB-WiP.csv} (which also
allows us to maintain the original file in case something goes wrong).

\hypertarget{pro-tip-1}{%
\subsection{Pro tip}\label{pro-tip-1}}

Always keep a backup copy of the data. Alternatively, set the data
file(s) to ``read-only'' to protect it from being overwritten or
modified.

\hypertarget{exercise}{%
\subsection{Exercise}\label{exercise}}

It is important to look at and understand the contents of the file
before you start using it. Using a text editor or a spreadsheet
software, open the \texttt{WB-WiP.csv} file (in the \texttt{data}
directory). What do you observe in the contents of this file?

\hypertarget{content-and-structure}{%
\subsection{Content and Structure}\label{content-and-structure}}

The first four lines of WB-WiP.csv can be ignored, since they contain
two lines of meta-information and two blank lines, as follows:

\begin{verbatim}
 1  "Data Source","World Development Indicators",
 2  
 3  "Last Updated Date","2019-03-21",
 4  
\end{verbatim}

The fifth line contains the column (variable) names and the body of data
starts in the sixth line. It is important to note that there was no
collection of data for a majority of the years, which means that it is
``missing''.

\hypertarget{importing-the-data}{%
\section{Importing the data}\label{importing-the-data}}

Based on our findings above, we can ``skip'' the first four lines and
treat the fifth line as column (variable) names.

\begin{Shaded}
\begin{Highlighting}[]
\KeywordTok{library}\NormalTok{(data.table)}
\KeywordTok{library}\NormalTok{(here)}
\NormalTok{wip <-}\StringTok{ }\KeywordTok{fread}\NormalTok{(}\KeywordTok{here}\NormalTok{(}\StringTok{"data"}\NormalTok{, }\StringTok{"WB-WiP.csv"}\NormalTok{), }
             \DataTypeTok{skip =} \DecValTok{4}\NormalTok{, }\DataTypeTok{header =} \OtherTok{TRUE}\NormalTok{)}
\end{Highlighting}
\end{Shaded}

\hypertarget{exercise-1}{%
\subsection{Exercise}\label{exercise-1}}

Check what you have read by typing ``\texttt{wip}'' in the console
window. What do you observe? Type ``\texttt{class(wip)}'' and
``\texttt{str(wip)}'' to confirm that ``\texttt{wip}'' is of class
``\texttt{data.table}''.

\hypertarget{fix-column-names}{%
\subsection{``Fix'' column names}\label{fix-column-names}}

Some of the column names contain spaces while others are numeric:

\begin{Shaded}
\begin{Highlighting}[]
\KeywordTok{head}\NormalTok{(}\KeywordTok{names}\NormalTok{(wip))}
\CommentTok{#  [1] "Country Name"   "Country Code"  }
\CommentTok{#  [3] "Indicator Name" "Indicator Code"}
\CommentTok{#  [5] "1960"           "1961"}
\KeywordTok{tail}\NormalTok{(}\KeywordTok{names}\NormalTok{(wip))}
\CommentTok{#  [1] "2014" "2015" "2016" "2017" "2018" "V64"}
\end{Highlighting}
\end{Shaded}

By using the \texttt{make.names()} function we don't need to use back
ticks (\texttt{\textasciigrave{}}) around the column names (e.g.
\texttt{\textasciigrave{}col\ name\textasciigrave{}}).

\begin{Shaded}
\begin{Highlighting}[]
\KeywordTok{names}\NormalTok{(wip) <-}\StringTok{ }\KeywordTok{make.names}\NormalTok{(}\KeywordTok{names}\NormalTok{(wip))}
\KeywordTok{head}\NormalTok{(}\KeywordTok{names}\NormalTok{(wip))}
\CommentTok{#  [1] "Country.Name"   "Country.Code"  }
\CommentTok{#  [3] "Indicator.Name" "Indicator.Code"}
\CommentTok{#  [5] "X1960"          "X1961"}
\KeywordTok{tail}\NormalTok{(}\KeywordTok{names}\NormalTok{(wip))}
\CommentTok{#  [1] "X2014" "X2015" "X2016" "X2017" "X2018"}
\CommentTok{#  [6] "V64"}
\end{Highlighting}
\end{Shaded}

\hypertarget{data-wrangling-aims}{%
\section{Data Wrangling Aims}\label{data-wrangling-aims}}

We can simplify the production of summaries and plots by restructuring
the current \texttt{wip} dataset (which has 64 columns) to the following
format:

\begin{verbatim}
   Country       Year   pctWiP
   Country AAA   1997     ##.#
   Country AAA   1998     ##.#
   Country AAA   1999     ##.#
   ...
\end{verbatim}

pctWiP refers to the percentage of women in parliament.

\hypertarget{key-information-retained}{%
\subsection{Key information retained}\label{key-information-retained}}

These three columns will contain the same information as the
\texttt{wip} dataset but in a more usable format. We will also add a
variable for the ratio of male to female MPs.

\hypertarget{superfluous-columns}{%
\subsection{Superfluous columns}\label{superfluous-columns}}

We will start by removing columns \texttt{V64}, \texttt{Indicator.Name}
and \texttt{Indicator.Code}. There are years without any data but they
will be removed automatically later (when restructuring from ``wide'' to
``long'' format).

Column \texttt{V64} is created automatically due to an extra comma at
the end of the column names (fifth) line of \texttt{WB-WiP.csv}:

\begin{verbatim}
   ... ,"2015","2016","2017","2018",
\end{verbatim}

\hypertarget{check}{%
\subsection{Check}\label{check}}

Before removing it check that all values are \texttt{NA}.

\begin{Shaded}
\begin{Highlighting}[]
\NormalTok{wip[, .N, by=.(V64)]}
\CommentTok{#     V64   N}
\CommentTok{#  1:  NA 264}
\end{Highlighting}
\end{Shaded}

Column \texttt{Indicator.Name} has the unique value \emph{``Proportion
of seats held by women in national parliaments (\%)''} and in
\texttt{Indicator.Code} it is \emph{``SG.GEN.PARL.ZS''}. As there is
only one indicator in this dataset we will remove these two columns.

\hypertarget{exercise-2}{%
\subsection{Exercise}\label{exercise-2}}

Confirm that both \texttt{Indicator.Name} and \texttt{Indicator.Code}
have the same values for all observations. Hint: Use the approach above
for variable \texttt{V64}

\hypertarget{removing-columns}{%
\subsection{Removing columns}\label{removing-columns}}

The indicator and \texttt{V64} columns can be removed. We will also
rename ``\texttt{Country.Name}'' as ``\texttt{Country}'' and
``\texttt{Country.Code}'' as ``\texttt{Code}''.

\begin{Shaded}
\begin{Highlighting}[]
\NormalTok{wip[, }\KeywordTok{c}\NormalTok{(}\StringTok{"Indicator.Name"}\NormalTok{, }\StringTok{"Indicator.Code"}\NormalTok{, }
        \StringTok{"V64"}\NormalTok{)}\OperatorTok{:}\ErrorTok{=}\OtherTok{NULL}\NormalTok{]}
\KeywordTok{setnames}\NormalTok{(wip, }\KeywordTok{c}\NormalTok{(}\StringTok{"Country.Name"}\NormalTok{, }\StringTok{"Country.Code"}\NormalTok{), }
              \KeywordTok{c}\NormalTok{(}\StringTok{"Country"}\NormalTok{, }\StringTok{"Code"}\NormalTok{))}
\KeywordTok{head}\NormalTok{(}\KeywordTok{names}\NormalTok{(wip))}
\CommentTok{#  [1] "Country" "Code"    "X1960"   "X1961"  }
\CommentTok{#  [5] "X1962"   "X1963"}
\KeywordTok{tail}\NormalTok{(}\KeywordTok{names}\NormalTok{(wip))}
\CommentTok{#  [1] "X2013" "X2014" "X2015" "X2016" "X2017"}
\CommentTok{#  [6] "X2018"}
\end{Highlighting}
\end{Shaded}

\hypertarget{reshape-to-long-format}{%
\subsection{Reshape to long format}\label{reshape-to-long-format}}

We want to transform the data so that for each country the year (column)
data becomes a row. At the same time we will remove the missing data
(with the \texttt{na.rm} option).

\begin{Shaded}
\begin{Highlighting}[]
\NormalTok{WP <-}\StringTok{ }\KeywordTok{melt}\NormalTok{(wip,}
           \DataTypeTok{id.vars =} \KeywordTok{c}\NormalTok{(}\StringTok{"Country"}\NormalTok{, }\StringTok{"Code"}\NormalTok{),}
           \DataTypeTok{measure =} \KeywordTok{patterns}\NormalTok{(}\StringTok{"^X"}\NormalTok{),}
           \DataTypeTok{variable.name =} \StringTok{"YearC"}\NormalTok{,}
           \DataTypeTok{value.name =} \KeywordTok{c}\NormalTok{(}\StringTok{"pctWiP"}\NormalTok{),}
           \DataTypeTok{na.rm =} \OtherTok{TRUE}\NormalTok{)}
\NormalTok{WP}
\CommentTok{#             Country Code YearC pctWiP}
\CommentTok{#     1:  Afghanistan  AFG X1990   3.70}
\CommentTok{#     2:       Angola  AGO X1990  14.50}
\CommentTok{#     3:      Albania  ALB X1990  28.80}
\CommentTok{#     4:   Arab World  ARB X1990   3.89}
\CommentTok{#    ---                               }
\CommentTok{#  5107:  Yemen, Rep.  YEM X2018   0.00}
\CommentTok{#  5108: South Africa  ZAF X2018  42.30}
\CommentTok{#  5109:       Zambia  ZMB X2018  18.00}
\CommentTok{#  5110:     Zimbabwe  ZWE X2018  31.50}
\end{Highlighting}
\end{Shaded}

\hypertarget{final-tweaks-to-wp}{%
\subsection{\texorpdfstring{Final tweaks to
\texttt{WP}}{Final tweaks to WP}}\label{final-tweaks-to-wp}}

Create a numeric \texttt{Year} variable and a \texttt{Ratio} of men to
women in parliament.

\begin{Shaded}
\begin{Highlighting}[]
\NormalTok{WP[, }\StringTok{`}\DataTypeTok{:=}\StringTok{`}\NormalTok{(}\DataTypeTok{Year=}\KeywordTok{as.numeric}\NormalTok{(}\KeywordTok{gsub}\NormalTok{(}\StringTok{"[^[:digit:].]"}\NormalTok{, }
                               \StringTok{""}\NormalTok{,  YearC)),}
          \DataTypeTok{Ratio =}\NormalTok{ (}\DecValTok{100}\OperatorTok{-}\NormalTok{pctWiP)}\OperatorTok{/}\NormalTok{pctWiP)][}
\NormalTok{            , YearC}\OperatorTok{:}\ErrorTok{=}\OtherTok{NULL}\NormalTok{]}
\KeywordTok{setcolorder}\NormalTok{(WP, }\KeywordTok{c}\NormalTok{(}\StringTok{"Country"}\NormalTok{, }\StringTok{"Code"}\NormalTok{, }\StringTok{"Year"}\NormalTok{, }
                  \StringTok{"pctWiP"}\NormalTok{, }\StringTok{"Ratio"}\NormalTok{))}
\CommentTok{# Look at the contents of WP}
\NormalTok{WP}
\CommentTok{#             Country Code Year pctWiP Ratio}
\CommentTok{#     1:  Afghanistan  AFG 1990   3.70 26.03}
\CommentTok{#     2:       Angola  AGO 1990  14.50  5.90}
\CommentTok{#     3:      Albania  ALB 1990  28.80  2.47}
\CommentTok{#     4:   Arab World  ARB 1990   3.89 24.70}
\CommentTok{#    ---                                    }
\CommentTok{#  5107:  Yemen, Rep.  YEM 2018   0.00   Inf}
\CommentTok{#  5108: South Africa  ZAF 2018  42.30  1.36}
\CommentTok{#  5109:       Zambia  ZMB 2018  18.00  4.56}
\CommentTok{#  5110:     Zimbabwe  ZWE 2018  31.50  2.17}
\end{Highlighting}
\end{Shaded}

\hypertarget{questions}{%
\section{Questions}\label{questions}}

The objective is to look at the geographical and time trends in the
data. We will answer the following questions.

\begin{itemize}
\tightlist
\item
  What are the time trends for Portugal?
\item
  How does Portugal compare to other countries?
\item
  Which countries have the highest percentage of women in parliament by
  year?
\item
  How do continents compare?
\item
  What are the global trends over time?
\end{itemize}

\hypertarget{exercise---without-programming}{%
\subsection{Exercise - Without
Programming}\label{exercise---without-programming}}

\begin{itemize}
\tightlist
\item
  Which country do you think has the highest percentage of women in
  parliament?
\item
  In each continent (i.e.~Africa, Americas, Asia, Europe and Oceania),
  which country has the highest percentage of women in parliament?
\item
  What is the world percentage of women in parliament in 2018?
\end{itemize}

\hypertarget{exploratory-analysis}{%
\section{Exploratory Analysis}\label{exploratory-analysis}}

\hypertarget{select-a-country}{%
\subsection{Select a country}\label{select-a-country}}

This guide explores how Portugal performs over time and compared to
other countries. Before continuing, select another country for yourself
to repeat the examples and do the exercises.

\hypertarget{time-trends-for-portugal}{%
\subsection{Time trends for Portugal}\label{time-trends-for-portugal}}

First look at the raw data.

\begin{Shaded}
\begin{Highlighting}[]
\NormalTok{WP[Country }\OperatorTok\StringTok{ "Portugal"}\NormalTok{]}
\CommentTok{#       Country Code Year pctWiP Ratio}
\CommentTok{#   1: Portugal  PRT 1990    7.6 12.16}
\CommentTok{#   2: Portugal  PRT 1997   13.0  6.69}
\CommentTok{#   3: Portugal  PRT 1998   13.0  6.69}
\CommentTok{#   4: Portugal  PRT 1999   18.7  4.35}
\CommentTok{#   5: Portugal  PRT 2000   17.4  4.75}
\CommentTok{#   6: Portugal  PRT 2001   18.7  4.35}
\CommentTok{#   7: Portugal  PRT 2002   19.1  4.24}
\CommentTok{#   8: Portugal  PRT 2003   19.1  4.24}
\CommentTok{#   9: Portugal  PRT 2004   19.1  4.24}
\CommentTok{#  10: Portugal  PRT 2005   21.3  3.69}
\CommentTok{#  11: Portugal  PRT 2006   21.3  3.69}
\CommentTok{#  12: Portugal  PRT 2007   28.3  2.53}
\CommentTok{#  13: Portugal  PRT 2008   28.3  2.53}
\CommentTok{#  14: Portugal  PRT 2009   27.4  2.65}
\CommentTok{#  15: Portugal  PRT 2010   27.4  2.65}
\CommentTok{#  16: Portugal  PRT 2011   28.7  2.48}
\CommentTok{#  17: Portugal  PRT 2012   28.7  2.48}
\CommentTok{#  18: Portugal  PRT 2013   28.7  2.48}
\CommentTok{#  19: Portugal  PRT 2014   31.3  2.19}
\CommentTok{#  20: Portugal  PRT 2015   34.8  1.87}
\CommentTok{#  21: Portugal  PRT 2016   34.8  1.87}
\CommentTok{#  22: Portugal  PRT 2017   34.8  1.87}
\CommentTok{#  23: Portugal  PRT 2018   34.8  1.87}
\CommentTok{#       Country Code Year pctWiP Ratio}
\end{Highlighting}
\end{Shaded}

\hypertarget{visualisation}{%
\subsubsection{Visualisation}\label{visualisation}}

It is easier to find trends within a plot.

\begin{Shaded}
\begin{Highlighting}[]
\KeywordTok{library}\NormalTok{(ggplot2)}
\KeywordTok{library}\NormalTok{(magrittr)}
\NormalTok{WP[Country }\OperatorTok\StringTok{ "Portugal"}\NormalTok{] }\OperatorTok\StringTok{ }
\KeywordTok{ggplot}\NormalTok{(}\KeywordTok{aes}\NormalTok{(Year, pctWiP)) }\OperatorTok{+}
\StringTok{  }\KeywordTok{geom_line}\NormalTok{() }\OperatorTok{+}\StringTok{ }\KeywordTok{geom_point}\NormalTok{() }\OperatorTok{+}
\StringTok{  }\KeywordTok{scale_y_continuous}\NormalTok{(}\DataTypeTok{limits=}\KeywordTok{c}\NormalTok{(}\DecValTok{0}\NormalTok{, }\DecValTok{50}\NormalTok{)) }\OperatorTok{+}
\StringTok{  }\KeywordTok{ylab}\NormalTok{(}\StringTok{"% Women in Parliament"}\NormalTok{)}
\end{Highlighting}
\end{Shaded}

\begin{center}\includegraphics{WiP-rdatatable_files/figure-latex/PTplot-1} \end{center}

\hypertarget{interpretation}{%
\subsubsection{Interpretation}\label{interpretation}}

In 1990 Portugal had 7.6\% women in parliament (i.e.~12.2 men for each
woman), which increased to 34.8\% (i.e.~1.87 men for each woman) in
2018. This still falls short of 50\% (i.e.~point of gender parity in
parliament).

\hypertarget{exercise-3}{%
\subsection{Exercise}\label{exercise-3}}

For your chosen country look at the time trend data and the plot. What
is your interpretation? How does it compare to Portugal?

\hypertarget{portugal-versus-european-union-eu-countries}{%
\subsection{Portugal versus European Union (EU)
countries}\label{portugal-versus-european-union-eu-countries}}

We selected six EU countries (due to space limitations) for comparison.
It would be better to compare all EU and/or all European countries.

\begin{Shaded}
\begin{Highlighting}[]
\NormalTok{WP[Country }\OperatorTok\StringTok{ }\KeywordTok{c}\NormalTok{(}\StringTok{"Portugal"}\NormalTok{, }\StringTok{"Sweden"}\NormalTok{, }\StringTok{"Spain"}\NormalTok{,}
     \StringTok{"Hungary"}\NormalTok{, }\StringTok{"Romania"}\NormalTok{, }\StringTok{"Finland"}\NormalTok{, }\StringTok{"Germany"}\NormalTok{,}
                           \StringTok{"European Union"}\NormalTok{)] }\OperatorTok
\StringTok{  }\KeywordTok{ggplot}\NormalTok{(}\KeywordTok{aes}\NormalTok{(Year, pctWiP, }\DataTypeTok{colour=}\NormalTok{Country)) }\OperatorTok{+}
\StringTok{  }\KeywordTok{geom_line}\NormalTok{() }\OperatorTok{+}
\StringTok{  }\KeywordTok{geom_point}\NormalTok{() }\OperatorTok{+}
\StringTok{  }\KeywordTok{scale_x_continuous}\NormalTok{(}\DataTypeTok{breaks=}\KeywordTok{seq}\NormalTok{(}\DecValTok{1990}\NormalTok{, }\DecValTok{2020}\NormalTok{, }\DecValTok{5}\NormalTok{)) }\OperatorTok{+}
\StringTok{  }\KeywordTok{scale_y_continuous}\NormalTok{(}\DataTypeTok{limits=}\KeywordTok{c}\NormalTok{(}\DecValTok{0}\NormalTok{, }\DecValTok{50}\NormalTok{), }
                     \DataTypeTok{breaks=}\KeywordTok{seq}\NormalTok{(}\DecValTok{0}\NormalTok{, }\DecValTok{50}\NormalTok{, }\DataTypeTok{by=}\DecValTok{10}\NormalTok{)) }\OperatorTok{+}
\StringTok{  }\KeywordTok{ggtitle}\NormalTok{(}\StringTok{"Women in Parliament: EU Countries"}\NormalTok{) }\OperatorTok{+}
\StringTok{  }\KeywordTok{ylab}\NormalTok{(}\StringTok{"% Women in Parliament"}\NormalTok{)}
\end{Highlighting}
\end{Shaded}

\begin{center}\includegraphics{WiP-rdatatable_files/figure-latex/euPctPlot-1} \end{center}

\hypertarget{interpretation-1}{%
\subsubsection{Interpretation}\label{interpretation-1}}

Since 2007 Portugal has had more women in parliament than the European
Union average. Hungary and Romania both had a higher percentage of women
in parliament in 1990 (around the end of the Cold War) than they have
had since. The key point to note is that none of these countries reaches
equality between males and females in parliament, although Sweden and
Finland come closest.

\hypertarget{a-couple-of-points-to-note}{%
\subsection{A couple of points to
note}\label{a-couple-of-points-to-note}}

\hypertarget{germany}{%
\subsubsection{``Germany''}\label{germany}}

In October 1990, the process of ``German reunification'' lead to the
creation of Germany, which united the former ``German Democratic
Republic'' (East Germany) and the ``Federal Republic of Germany'' (West
Germany). Therefore, since reunification, the data is presented for the
reunified ``Germany'' only. Careful thought should be given to handling,
analysing and interpreting any pre-reunification data (if available).

\hypertarget{european-union}{%
\subsubsection{``European Union''}\label{european-union}}

The ``European Union'' has changed over time (unlike the ``continent of
Europe''). It started in the 1950s as a block of six European countries
(known as the ``European Community'') and has expanded over the years to
28 countries (with the United Kingdom about to depart). This raises the
question of how the European Union average is calculated. For a given
year, is it calculated based on the actual member states in that year or
on all of the current member states?

\hypertarget{exercises}{%
\subsection{Exercises}\label{exercises}}

Compare the country of your choice to four or five other countries by
plotting a line graph similar to the one above.

\hypertarget{countries-with-the-highest-percentage-of-women-in-parliament}{%
\subsection{Countries with the highest percentage of women in
parliament}\label{countries-with-the-highest-percentage-of-women-in-parliament}}

A quick answer can be obtained by looking at the highest percentages.

\begin{Shaded}
\begin{Highlighting}[]
\NormalTok{WP[}\KeywordTok{order}\NormalTok{(}\OperatorTok{-}\NormalTok{pctWiP), }\KeywordTok{head}\NormalTok{(.SD, }\DecValTok{10}\NormalTok{)]}
\CommentTok{#      Country Code Year pctWiP Ratio}
\CommentTok{#   1:  Rwanda  RWA 2013   63.8 0.567}
\CommentTok{#   2:  Rwanda  RWA 2014   63.8 0.567}
\CommentTok{#   3:  Rwanda  RWA 2015   63.8 0.567}
\CommentTok{#   4:  Rwanda  RWA 2016   63.8 0.567}
\CommentTok{#   5:  Rwanda  RWA 2017   61.3 0.631}
\CommentTok{#   6:  Rwanda  RWA 2018   61.3 0.631}
\CommentTok{#   7:  Rwanda  RWA 2008   56.3 0.776}
\CommentTok{#   8:  Rwanda  RWA 2009   56.3 0.776}
\CommentTok{#   9:  Rwanda  RWA 2010   56.3 0.776}
\CommentTok{#  10:  Rwanda  RWA 2011   56.3 0.776}
\end{Highlighting}
\end{Shaded}

\hypertarget{data-speaks}{%
\subsection{Data speaks}\label{data-speaks}}

Are you surprised? Data can be very enlightening.

\hypertarget{highest-percentage-by-year}{%
\subsection{Highest percentage by
year}\label{highest-percentage-by-year}}

Which countries have the highest percentage of women in parliament by
year?

\begin{Shaded}
\begin{Highlighting}[]
\NormalTok{WP[}\KeywordTok{order}\NormalTok{(Year, }\OperatorTok{-}\NormalTok{pctWiP), }\KeywordTok{head}\NormalTok{(.SD, }\DecValTok{1}\NormalTok{), by =}\StringTok{ }\NormalTok{Year]}
\CommentTok{#      Year Country Code pctWiP Ratio}
\CommentTok{#   1: 1990  Sweden  SWE   38.4 1.604}
\CommentTok{#   2: 1997  Sweden  SWE   40.4 1.475}
\CommentTok{#   3: 1998  Sweden  SWE   40.4 1.475}
\CommentTok{#   4: 1999  Sweden  SWE   42.7 1.342}
\CommentTok{#   5: 2000  Sweden  SWE   42.7 1.342}
\CommentTok{#   6: 2001  Sweden  SWE   42.7 1.342}
\CommentTok{#   7: 2002  Sweden  SWE   45.0 1.222}
\CommentTok{#   8: 2003  Rwanda  RWA   48.8 1.049}
\CommentTok{#   9: 2004  Rwanda  RWA   48.8 1.049}
\CommentTok{#  10: 2005  Rwanda  RWA   48.8 1.049}
\CommentTok{#  11: 2006  Rwanda  RWA   48.8 1.049}
\CommentTok{#  12: 2007  Rwanda  RWA   48.8 1.049}
\CommentTok{#  13: 2008  Rwanda  RWA   56.3 0.776}
\CommentTok{#  14: 2009  Rwanda  RWA   56.3 0.776}
\CommentTok{#  15: 2010  Rwanda  RWA   56.3 0.776}
\CommentTok{#  16: 2011  Rwanda  RWA   56.3 0.776}
\CommentTok{#  17: 2012  Rwanda  RWA   56.3 0.776}
\CommentTok{#  18: 2013  Rwanda  RWA   63.8 0.567}
\CommentTok{#  19: 2014  Rwanda  RWA   63.8 0.567}
\CommentTok{#  20: 2015  Rwanda  RWA   63.8 0.567}
\CommentTok{#  21: 2016  Rwanda  RWA   63.8 0.567}
\CommentTok{#  22: 2017  Rwanda  RWA   61.3 0.631}
\CommentTok{#  23: 2018  Rwanda  RWA   61.3 0.631}
\CommentTok{#      Year Country Code pctWiP Ratio}
\end{Highlighting}
\end{Shaded}

\hypertarget{merging-continent}{%
\subsection{Merging continent}\label{merging-continent}}

The variable \texttt{Country} in the \texttt{WP} dataset is a mix of
countries and regions (e.g. ``European Union'', ``South Asia'' and
``World''). To present the highest percentages grouped by continent we
need to add it. Luckily, given the large number of R packages available,
we can merge the ``\texttt{continent}'' from the ``\texttt{codelist}''
dataset in the ``\texttt{countrycode}'' package.

\begin{Shaded}
\begin{Highlighting}[]
\CommentTok{# Ensure that 'countrycode' package is installed.}
\CommentTok{# install.packages("countrycode")}
\KeywordTok{library}\NormalTok{(countrycode)}
\NormalTok{cl <-}\StringTok{ }\KeywordTok{as.data.table}\NormalTok{(codelist)[, .(continent, wb)]}
\KeywordTok{setnames}\NormalTok{(cl, }\KeywordTok{c}\NormalTok{(}\StringTok{"continent"}\NormalTok{, }\StringTok{"wb"}\NormalTok{), }
             \KeywordTok{c}\NormalTok{(}\StringTok{"Continent"}\NormalTok{, }\StringTok{"Code"}\NormalTok{))}
\NormalTok{cWP <-}\StringTok{ }\NormalTok{cl[WP, on=}\StringTok{"Code"}\NormalTok{]}
\end{Highlighting}
\end{Shaded}

\hypertarget{highest-percentages-by-year-and-continent}{%
\subsection{Highest percentages by year and
continent}\label{highest-percentages-by-year-and-continent}}

Which countries have the highest percentages in 1990 and 2018?

\begin{Shaded}
\begin{Highlighting}[]
\NormalTok{cWP[Year }\OperatorTok\StringTok{ }\KeywordTok{c}\NormalTok{(}\DecValTok{1990}\NormalTok{, }\DecValTok{2018}\NormalTok{) }\OperatorTok{&}\StringTok{ }\OperatorTok{!}\KeywordTok{is.na}\NormalTok{(Continent)][}
    \KeywordTok{order}\NormalTok{(Year, }\OperatorTok{-}\NormalTok{pctWiP), }\KeywordTok{head}\NormalTok{(.SD, }\DecValTok{1}\NormalTok{), }
\NormalTok{    by =}\StringTok{ }\NormalTok{.(Year, Continent)][}
    \KeywordTok{order}\NormalTok{(Continent, Year), }
\NormalTok{    .(Continent, Year, Country, pctWiP)]}
\CommentTok{#      Continent Year       Country pctWiP}
\CommentTok{#   1:    Africa 1990 Guinea-Bissau   20.0}
\CommentTok{#   2:    Africa 2018        Rwanda   61.3}
\CommentTok{#   3:  Americas 1990        Guyana   36.9}
\CommentTok{#   4:  Americas 2018          Cuba   53.2}
\CommentTok{#   5:      Asia 1990       Armenia   35.6}
\CommentTok{#   6:      Asia 2018   Timor-Leste   33.8}
\CommentTok{#   7:    Europe 1990        Sweden   38.4}
\CommentTok{#   8:    Europe 2018        Sweden   46.1}
\CommentTok{#   9:   Oceania 1990   New Zealand   14.4}
\CommentTok{#  10:   Oceania 2018   New Zealand   38.3}
\end{Highlighting}
\end{Shaded}

\hypertarget{decline-in-percentage}{%
\subsection{Decline in percentage}\label{decline-in-percentage}}

Which countries have had a decline in percentage since their first
measurement (not always 1990)?

\begin{Shaded}
\begin{Highlighting}[]
\NormalTok{dWP <-}\StringTok{ }\NormalTok{cWP[}
  \KeywordTok{order}\NormalTok{(Country, Year), .SD[}\KeywordTok{c}\NormalTok{(}\DecValTok{1}\NormalTok{,.N)], }
\NormalTok{   by=Country][,}
\NormalTok{  pctDiff }\OperatorTok{:}\ErrorTok{=}\StringTok{ }\NormalTok{pctWiP }\OperatorTok{-}\StringTok{ }\KeywordTok{shift}\NormalTok{(pctWiP), by=Country][}
\NormalTok{  pctDiff}\OperatorTok{<}\DecValTok{0}\NormalTok{][}
  \KeywordTok{order}\NormalTok{(pctDiff)]}
\NormalTok{dWP[}\OperatorTok{!}\KeywordTok{is.na}\NormalTok{(Continent),}
\NormalTok{    .(Country, pctWiP, pctDiff)]}
\CommentTok{#                        Country pctWiP pctDiff}
\CommentTok{#   1:                   Armenia   18.1   -17.5}
\CommentTok{#   2:                   Romania   20.7   -13.7}
\CommentTok{#   3:                   Hungary   12.6    -8.1}
\CommentTok{#   4:                  Mongolia   17.1    -7.8}
\CommentTok{#   5:             Guinea-Bissau   13.7    -6.3}
\CommentTok{#   6:                    Guyana   31.9    -5.0}
\CommentTok{#   7: Korea, Dem. People’s Rep.   16.3    -4.8}
\CommentTok{#   8:                   Vanuatu    0.0    -4.3}
\CommentTok{#   9:               Yemen, Rep.    0.0    -4.1}
\CommentTok{#  10:                      Mali    8.8    -3.4}
\CommentTok{#  11:               Congo, Rep.   11.3    -3.0}
\CommentTok{#  12:                      Oman    1.2    -1.2}
\CommentTok{#  13:              Turkmenistan   24.8    -1.2}
\CommentTok{#  14:                     Haiti    2.5    -1.1}
\CommentTok{#  15:                    Tuvalu    6.7    -1.0}
\CommentTok{#  16:                   Albania   27.9    -0.9}
\CommentTok{#  17:                  Maldives    5.9    -0.4}
\end{Highlighting}
\end{Shaded}

\hypertarget{visualisation-1}{%
\subsubsection{Visualisation}\label{visualisation-1}}

We will plot the trend lines for countries with at least a 5\% decline.
Note that the ``5\%'' is arbitrarily selected.

\begin{Shaded}
\begin{Highlighting}[]
\CommentTok{# Select the countries to plot}
\NormalTok{dclpct <-}\StringTok{ }\KeywordTok{unique}\NormalTok{(dWP[}\OperatorTok{!}\KeywordTok{is.na}\NormalTok{(Continent) }\OperatorTok{&}
\StringTok{                   }\NormalTok{pctDiff }\OperatorTok{<=}\StringTok{ }\DecValTok{-5}\NormalTok{]}\OperatorTok{$}\NormalTok{Country)}

\NormalTok{WP[Country }\OperatorTok\StringTok{ }\NormalTok{dclpct] }\OperatorTok
\StringTok{  }\KeywordTok{ggplot}\NormalTok{(}\KeywordTok{aes}\NormalTok{(Year, pctWiP, }\DataTypeTok{colour=}\NormalTok{Country)) }\OperatorTok{+}
\StringTok{  }\KeywordTok{geom_line}\NormalTok{() }\OperatorTok{+}
\StringTok{  }\KeywordTok{geom_point}\NormalTok{() }\OperatorTok{+}
\StringTok{  }\KeywordTok{scale_x_continuous}\NormalTok{(}\DataTypeTok{breaks=}\KeywordTok{seq}\NormalTok{(}\DecValTok{1990}\NormalTok{, }\DecValTok{2020}\NormalTok{, }\DecValTok{5}\NormalTok{)) }\OperatorTok{+}
\StringTok{  }\KeywordTok{scale_y_continuous}\NormalTok{(}\DataTypeTok{limits=}\KeywordTok{c}\NormalTok{(}\DecValTok{0}\NormalTok{, }\DecValTok{40}\NormalTok{),}
  \DataTypeTok{breaks=}\KeywordTok{seq}\NormalTok{(}\DecValTok{0}\NormalTok{, }\DecValTok{40}\NormalTok{, }\DataTypeTok{by=}\DecValTok{10}\NormalTok{)) }\OperatorTok{+}
\StringTok{  }\KeywordTok{ggtitle}\NormalTok{(}\StringTok{"Women in Parliament: Decline >=5%"}\NormalTok{) }\OperatorTok{+}
\StringTok{  }\KeywordTok{ylab}\NormalTok{(}\StringTok{"% Women in Parliament"}\NormalTok{)}
\end{Highlighting}
\end{Shaded}

\begin{center}\includegraphics{WiP-rdatatable_files/figure-latex/decline5pct-1} \end{center}

\hypertarget{interpretation-2}{%
\subsubsection{Interpretation}\label{interpretation-2}}

There is a consistent decline between 1990 and 1997 that should be
investigated in collaboration with a subject matter expert to understand
the potential causes.

\hypertarget{ranked-status}{%
\subsection{Ranked status}\label{ranked-status}}

Another way to look at the data is to look at the ranking of countries,
which could be done at a global level or by continent. Nonetheless, the
results should be interpreted with caution and an understanding of the
actual percentages. For example, if most countries were around the 50\%
mark, rankings could be misleading and subject to random fluctuations.

\hypertarget{global-ranks-by-year}{%
\subsection{Global ranks by year}\label{global-ranks-by-year}}

We will rank the countries by year based on the percentage of women in
parliaments. The countries with the highest percentage will be ranked
first and the lowest last. A total for the number of countries with data
is included as it varies by year.

\begin{Shaded}
\begin{Highlighting}[]
\NormalTok{cWP[}\OperatorTok{!}\KeywordTok{is.na}\NormalTok{(Continent), }
    \StringTok{`}\DataTypeTok{:=}\StringTok{`}\NormalTok{(}\DataTypeTok{RankG =} \KeywordTok{rank}\NormalTok{(}\OperatorTok{-}\NormalTok{pctWiP), }\DataTypeTok{TotalG =}\NormalTok{ .N),}
\NormalTok{        by =}\StringTok{ }\NormalTok{.(Year)]}
\end{Highlighting}
\end{Shaded}

\hypertarget{global-ranking-portugal}{%
\subsection{Global ranking -- Portugal}\label{global-ranking-portugal}}

\begin{Shaded}
\begin{Highlighting}[]
\NormalTok{cWP[Country}\OperatorTok{==}\StringTok{"Portugal"}\NormalTok{, }
\NormalTok{  .(Country, Year, pctWiP, Ratio, RankG, TotalG)][}
  \KeywordTok{order}\NormalTok{(Year)]}
\CommentTok{#       Country Year pctWiP Ratio RankG TotalG}
\CommentTok{#   1: Portugal 1990    7.6 12.16  66.0    138}
\CommentTok{#   2: Portugal 1997   13.0  6.69  41.5    161}
\CommentTok{#   3: Portugal 1998   13.0  6.69  49.5    163}
\CommentTok{#   4: Portugal 1999   18.7  4.35  24.0    154}
\CommentTok{#   5: Portugal 2000   17.4  4.75  34.0    158}
\CommentTok{#   6: Portugal 2001   18.7  4.35  33.0    168}
\CommentTok{#   7: Portugal 2002   19.1  4.24  41.0    161}
\CommentTok{#   8: Portugal 2003   19.1  4.24  46.0    175}
\CommentTok{#   9: Portugal 2004   19.1  4.24  54.0    182}
\CommentTok{#  10: Portugal 2005   21.3  3.69  45.5    185}
\CommentTok{#  11: Portugal 2006   21.3  3.69  49.5    189}
\CommentTok{#  12: Portugal 2007   28.3  2.53  28.0    188}
\CommentTok{#  13: Portugal 2008   28.3  2.53  28.0    187}
\CommentTok{#  14: Portugal 2009   27.4  2.65  33.0    187}
\CommentTok{#  15: Portugal 2010   27.4  2.65  34.0    187}
\CommentTok{#  16: Portugal 2011   28.7  2.48  31.0    188}
\CommentTok{#  17: Portugal 2012   28.7  2.48  35.0    188}
\CommentTok{#  18: Portugal 2013   28.7  2.48  39.0    186}
\CommentTok{#  19: Portugal 2014   31.3  2.19  35.0    187}
\CommentTok{#  20: Portugal 2015   34.8  1.87  29.0    188}
\CommentTok{#  21: Portugal 2016   34.8  1.87  27.0    191}
\CommentTok{#  22: Portugal 2017   34.8  1.87  28.0    192}
\CommentTok{#  23: Portugal 2018   34.8  1.87  29.0    192}
\CommentTok{#       Country Year pctWiP Ratio RankG TotalG}
\end{Highlighting}
\end{Shaded}

\hypertarget{interpretation-3}{%
\subsubsection{Interpretation}\label{interpretation-3}}

Portugal has generally been ranked in the first quartile (25\%) of
countries in the world, with the fluctuations of its ranking most likely
due to random variation.

\hypertarget{exercise-4}{%
\subsection{Exercise}\label{exercise-4}}

For your chosen country, interpret its ranking over the years. How does
it compare to Portugal?

\hypertarget{continent-ranks-by-year}{%
\subsection{Continent ranks by year}\label{continent-ranks-by-year}}

We will rank the countries by year within a continent based on the
percentage of women in parliaments. The countries with the highest
percentage will be ranked first and the lowest last. A total for the
number of countries with data, within each continent, is included as it
varies by year.

\begin{Shaded}
\begin{Highlighting}[]
\NormalTok{cWP[}\OperatorTok{!}\KeywordTok{is.na}\NormalTok{(Continent), }
    \StringTok{`}\DataTypeTok{:=}\StringTok{`}\NormalTok{(}\DataTypeTok{RankC =} \KeywordTok{rank}\NormalTok{(}\OperatorTok{-}\NormalTok{pctWiP), }\DataTypeTok{TotalC =}\NormalTok{ .N),}
\NormalTok{        by =}\StringTok{ }\NormalTok{.(Continent, Year)]}
\end{Highlighting}
\end{Shaded}

\hypertarget{portugals-ranking-in-europe}{%
\subsection{Portugal's ranking in
Europe}\label{portugals-ranking-in-europe}}

\begin{Shaded}
\begin{Highlighting}[]
\NormalTok{cWP[Country}\OperatorTok{==}\StringTok{"Portugal"}\NormalTok{, }
\NormalTok{  .(Country, Year, pctWiP, Ratio, RankC, TotalC)][}
  \KeywordTok{order}\NormalTok{(Year)]}
\CommentTok{#       Country Year pctWiP Ratio RankC TotalC}
\CommentTok{#   1: Portugal 1990    7.6 12.16  22.0     28}
\CommentTok{#   2: Portugal 1997   13.0  6.69  16.5     39}
\CommentTok{#   3: Portugal 1998   13.0  6.69  18.5     37}
\CommentTok{#   4: Portugal 1999   18.7  4.35  12.0     38}
\CommentTok{#   5: Portugal 2000   17.4  4.75  16.0     38}
\CommentTok{#   6: Portugal 2001   18.7  4.35  15.0     41}
\CommentTok{#   7: Portugal 2002   19.1  4.24  15.0     39}
\CommentTok{#   8: Portugal 2003   19.1  4.24  17.0     41}
\CommentTok{#   9: Portugal 2004   19.1  4.24  21.0     42}
\CommentTok{#  10: Portugal 2005   21.3  3.69  20.0     42}
\CommentTok{#  11: Portugal 2006   21.3  3.69  21.0     44}
\CommentTok{#  12: Portugal 2007   28.3  2.53  15.0     44}
\CommentTok{#  13: Portugal 2008   28.3  2.53  13.0     44}
\CommentTok{#  14: Portugal 2009   27.4  2.65  15.0     44}
\CommentTok{#  15: Portugal 2010   27.4  2.65  15.0     44}
\CommentTok{#  16: Portugal 2011   28.7  2.48  14.0     44}
\CommentTok{#  17: Portugal 2012   28.7  2.48  15.0     43}
\CommentTok{#  18: Portugal 2013   28.7  2.48  17.0     43}
\CommentTok{#  19: Portugal 2014   31.3  2.19  16.0     43}
\CommentTok{#  20: Portugal 2015   34.8  1.87  12.0     43}
\CommentTok{#  21: Portugal 2016   34.8  1.87  11.0     43}
\CommentTok{#  22: Portugal 2017   34.8  1.87  12.0     43}
\CommentTok{#  23: Portugal 2018   34.8  1.87  12.0     43}
\CommentTok{#       Country Year pctWiP Ratio RankC TotalC}
\end{Highlighting}
\end{Shaded}

\hypertarget{plot-of-portugals-ranking-in-europe}{%
\subsection{Plot of Portugal's ranking in
Europe}\label{plot-of-portugals-ranking-in-europe}}

Below we reproduce the percentage plot to show how Portugal ranks in
relation to six other European countries. Note that the highest
percentage is ranked first and the lowest last.

\begin{Shaded}
\begin{Highlighting}[]
\NormalTok{cWP[Country }\OperatorTok\StringTok{ }\KeywordTok{c}\NormalTok{(}\StringTok{"Portugal"}\NormalTok{, }\StringTok{"Sweden"}\NormalTok{, }\StringTok{"Spain"}\NormalTok{,}
  \StringTok{"Hungary"}\NormalTok{, }\StringTok{"Romania"}\NormalTok{, }\StringTok{"Finland"}\NormalTok{, }\StringTok{"Germany"}\NormalTok{)] }\OperatorTok
\StringTok{  }\KeywordTok{ggplot}\NormalTok{(}\KeywordTok{aes}\NormalTok{(Year, RankC, }\DataTypeTok{colour=}\NormalTok{Country)) }\OperatorTok{+}
\StringTok{  }\KeywordTok{geom_line}\NormalTok{() }\OperatorTok{+}
\StringTok{  }\KeywordTok{geom_point}\NormalTok{() }\OperatorTok{+}
\StringTok{  }\KeywordTok{scale_x_continuous}\NormalTok{(}\DataTypeTok{breaks=}\KeywordTok{seq}\NormalTok{(}\DecValTok{1990}\NormalTok{, }\DecValTok{2020}\NormalTok{, }\DecValTok{5}\NormalTok{)) }\OperatorTok{+}
\StringTok{  }\KeywordTok{scale_y_continuous}\NormalTok{(}\DataTypeTok{limits=}\KeywordTok{c}\NormalTok{(}\DecValTok{0}\NormalTok{, }\DecValTok{45}\NormalTok{), }
                     \DataTypeTok{breaks=}\KeywordTok{seq}\NormalTok{(}\DecValTok{0}\NormalTok{, }\DecValTok{45}\NormalTok{, }\DataTypeTok{by=}\DecValTok{10}\NormalTok{)) }\OperatorTok{+}
\StringTok{  }\KeywordTok{ggtitle}\NormalTok{(}\StringTok{"Women in Parliament: Ranked"}\NormalTok{) }\OperatorTok{+}
\StringTok{  }\KeywordTok{ylab}\NormalTok{(}\StringTok{"Rank in Europe"}\NormalTok{)}
\end{Highlighting}
\end{Shaded}

\begin{center}\includegraphics{WiP-rdatatable_files/figure-latex/euRankplot-1} \end{center}

\hypertarget{interpretation-4}{%
\subsubsection{Interpretation}\label{interpretation-4}}

A total of 28 European countries had data in 1990, 39 in 1997 and 43 in
2018. Within Europe, Portugal was typically ranked in the second
quartile (25-50\%) with the fluctuations of its ranking most likely due
to random variation.

\hypertarget{exercise-5}{%
\subsection{Exercise}\label{exercise-5}}

How does your chosen country rank within its continent?

\hypertarget{highest-rank-by-year-and-continent}{%
\subsection{Highest rank by year and
continent}\label{highest-rank-by-year-and-continent}}

Which countries have the highest rank in 1990 and 2018? The answer will
coincide with the highest percentages (see above).

\begin{Shaded}
\begin{Highlighting}[]
\NormalTok{cWP[Year }\OperatorTok\StringTok{ }\KeywordTok{c}\NormalTok{(}\DecValTok{1990}\NormalTok{, }\DecValTok{2018}\NormalTok{) }\OperatorTok{&}\StringTok{ }\NormalTok{RankC}\OperatorTok{==}\DecValTok{1}\NormalTok{][}
    \KeywordTok{order}\NormalTok{(Continent, Year), }
\NormalTok{      .(Continent, Year, Country, pctWiP, RankC)]}
\CommentTok{#   Continent Year       Country pctWiP RankC}
\CommentTok{#      Africa 1990 Guinea-Bissau   20.0     1}
\CommentTok{#      Africa 2018        Rwanda   61.3     1}
\CommentTok{#    Americas 1990        Guyana   36.9     1}
\CommentTok{#    Americas 2018          Cuba   53.2     1}
\CommentTok{#        Asia 1990       Armenia   35.6     1}
\CommentTok{#        Asia 2018   Timor-Leste   33.8     1}
\CommentTok{#      Europe 1990        Sweden   38.4     1}
\CommentTok{#      Europe 2018        Sweden   46.1     1}
\CommentTok{#     Oceania 1990   New Zealand   14.4     1}
\CommentTok{#     Oceania 2018   New Zealand   38.3     1}
\end{Highlighting}
\end{Shaded}

\hypertarget{overall-picture}{%
\subsection{Overall picture}\label{overall-picture}}

What are the trends globally? There are various regions defined in the
World Bank data. We can plot them and highlight the world ``average''.

\begin{Shaded}
\begin{Highlighting}[]
\KeywordTok{library}\NormalTok{(gghighlight)}
\NormalTok{cWP[}\KeywordTok{is.na}\NormalTok{(Continent)] }\OperatorTok
\StringTok{  }\KeywordTok{ggplot}\NormalTok{(}\KeywordTok{aes}\NormalTok{(Year, pctWiP, }\DataTypeTok{group=}\NormalTok{Country)) }\OperatorTok{+}
\StringTok{  }\KeywordTok{geom_line}\NormalTok{() }\OperatorTok{+}
\StringTok{  }\KeywordTok{gghighlight}\NormalTok{(Country}\OperatorTok{==}\StringTok{"World"}\NormalTok{, }
              \DataTypeTok{use_direct_label =} \OtherTok{FALSE}\NormalTok{) }\OperatorTok{+}
\StringTok{  }\KeywordTok{scale_x_continuous}\NormalTok{(}\DataTypeTok{breaks=}\KeywordTok{seq}\NormalTok{(}\DecValTok{1990}\NormalTok{, }\DecValTok{2020}\NormalTok{, }\DecValTok{5}\NormalTok{)) }\OperatorTok{+}
\StringTok{  }\KeywordTok{scale_y_continuous}\NormalTok{(}\DataTypeTok{limits=}\KeywordTok{c}\NormalTok{(}\DecValTok{0}\NormalTok{, }\DecValTok{40}\NormalTok{), }
                     \DataTypeTok{breaks=}\KeywordTok{seq}\NormalTok{(}\DecValTok{0}\NormalTok{, }\DecValTok{40}\NormalTok{, }\DataTypeTok{by=}\DecValTok{10}\NormalTok{)) }\OperatorTok{+}
\StringTok{  }\KeywordTok{ggtitle}\NormalTok{(}\StringTok{"Women in Parliament: Global Trends"}\NormalTok{) }\OperatorTok{+}
\StringTok{  }\KeywordTok{ylab}\NormalTok{(}\StringTok{"% Women in Parliament"}\NormalTok{)}
\end{Highlighting}
\end{Shaded}

\begin{center}\includegraphics{WiP-rdatatable_files/figure-latex/globalTrends-1} \end{center}

\hypertarget{interpretation-5}{%
\subsubsection{Interpretation}\label{interpretation-5}}

The grey lines show that regardless of how we define region the general
trends are upwards. The ``World'' percentage (black line) increased
between 1997 and 2018. In 2018, women in parliament represented 24\%
(i.e.~a ratio of 3.17 men to each woman), which is still less than half
the level before gender parity can be claimed.

\hypertarget{conclusion}{%
\section{Conclusion}\label{conclusion}}

This guide presented an analysis of the percentage of women in
parliament as a real-life case study for the data.table package.
Although the format limited what could be presented, we can conclude
that the percentage of women in parliament is increasing but that gender
parity in parliaments is still far-off.

There is a lot more that can be said and discussed about the
limitations, interpretation and potential impact of this data which the
World Bank has nicely
\href{https://databank.worldbank.org/data/reports.aspx?source=2\&type=metadata\&series=SG.GEN.PARL.ZS\#}{summarised}.\footnote{\url{https://databank.worldbank.org/data/reports.aspx?source=2\&type=metadata\&series=SG.GEN.PARL.ZS}.}
You are strongly encouraged to read their discussion for a more complete
understanding.

%\showmatmethods


\bibliography{pinp}
\bibliographystyle{jss}



\end{document}

